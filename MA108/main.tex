\documentclass{article}
\usepackage[utf8]{inputenc}
\usepackage{amssymb}
\usepackage{amsmath}
\addtolength{\oddsidemargin}{-.875in}
\addtolength{\evensidemargin}{-.875in}
\addtolength{\textwidth}{1.75in}

\addtolength{\topmargin}{-1in}
\addtolength{\textheight}{1.75in}

\usepackage{graphicx}
\graphicspath{ {./Desktop/Notes/MA108} }

\title{MA108: Ordinary Differential Equations}
\author{Shubh Kumar}
\date{IIT-B, Spring Semester 2021}

\begin{document}

\maketitle

\section{Lecture 1: Introduction}

  \begin{itemize}
    \item Ordinary DEs mean those not involving Partial Deriavatives
    \item Can Solutions be expressed in a nice form? : Can we have a few linearly independent solutions which together span the space of all solutions?
    \item What combinations of the basic solutions of the form: \\
    \begin{center}
      $a_{n}(x)y^{(n)}+a_{n-1}y^{(n-1)}....a_{0}y = b(x)$ \\
      will also be a solution?
    \end{center}
  \end{itemize}

 \section {Lecture 2: Seperable and Homogenous Equations}

 \begin{itemize}
   \item At each point $(a,b)$ if we define a vector field, $H: D \rightarrow R^{2}$ given by \\
   \begin{center}
     $H(x,y) = (1,f(a,b))$ \\~\\
     provided $\frac{dy}{dx} = f(x,y)$ \\
   \end{center}
   In this case, What can you do to find a solution curve provided this plotting of the said vector field?

   \item
 \end{itemize}


\end{document}
