\documentclass{article}
\usepackage[utf8]{inputenc}
\usepackage{graphicx}

\addtolength{\oddsidemargin}{-.875in}
\addtolength{\evensidemargin}{-.875in}
\addtolength{\textwidth}{1.75in}
\usepackage{amsmath}
\addtolength{\topmargin}{-1in}
\addtolength{\textheight}{1.75in}


\title{PH108}
\author{Shubh Kumar}
\date{IITB, Spring Semester 2021}

\begin{document}

\maketitle

\section{Lecture 1 : Introduction}

\begin{itemize}
    \item G and K, the constants of Gravitation and Electrostatic Forces are called Coupling Constants when taken together

    \item A Physical Law is one which is true for observers from all \textit{Frames of References}.

    \item Strong Force holds the Nucleus Together. Weak Force is the one responsible for Radioactivity
\end{itemize}

\section{Formula from Recorded Lecture}
\begin{itemize}
    \item Pertaining to the $\nabla$ operator:
        \begin{itemize}
            \item $\nabla (fg) = f\nabla g + g\nabla f$
            \item $\nabla (\vec{A} \cdot \vec{B}) = \vec{A} \times ( \nabla \times \vec{B}) + \vec{B} \times ( \nabla \times \vec{A})+
            (\nabla \cdot \vec{A})\vec{B} + (\nabla \cdot \vec{B})\vec{A}$
        \end{itemize}




\item Those Pertaining to Divergence ($\nabla \cdot$):
        \begin{itemize}
            \item $\nabla \cdot (f\vec{A}) = f(\nabla \cdot \vec{A}) + \vec{A} \cdot (\nabla f)$
            \item $\nabla \cdot (\vec{A} \times \vec{B}) = \vec{B} \cdot (\nabla \times \vec{A})-\vec{A} \cdot (\nabla \times \vec{B})$
        \end{itemize}
\item Those Pertaining to Curl ($\nabla \times$):
\begin{itemize}
    \item $\nabla \times (f\vec{A}) = f(\nabla \times \vec{A}) - \vec{A} \times (\nabla f)$
    \item $\nabla \times (\vec{A} \times \vec{B}) = (\vec{B}. \nabla)\vec{A} + \vec{A} (\nabla \cdot \vec{B})-(\vec{A}. \nabla)\vec{B}-\vec{B} (\nabla \cdot \vec{A})$
\end{itemize}

\item \textbf{Curl of Curl:} $\nabla (\nabla \cdot \vec{A})- \nabla^{2}\vec{A}$
\end{itemize}

\section{Lecture 2: }
Nothing Extra as such, apart form Lecture Slides!

\section{Recorded Lecture 2: Flux, Gauss's Divergence Theorem, Stokes', Curvilinear Co-ordinates in a plane}
\begin{itemize}
  \item The $\hat{r}$ depends implicity on $\theta$. (Although not explicitly!)
  \item Similar to the way in which, we used to take the norm of the area vector by taking its deriavtive with the numbers, which
  define it, we could do something similar with length as well!
  \item The $\nabla$ operator takes a different form when we consider Curvilinear co-ordinates in a plane,as conventionally its defined as the change of the scalar function w.r.t the changes in the x and y co-ordinates only, such that when we take a line-integral taking the distance appropriately in Curvilinear co-ordinates then we could get the right answer.
  \item \textbf{Formulae:}
  \begin{itemize}
    \item $\begin{pmatrix}
              \hat{r} \\
              \hat{\theta}
          \end{pmatrix} =
          \begin{pmatrix}
            \cos (\theta) && \sin (\theta) \\
            -\sin (\theta) && \cos (\theta)
          \end{pmatrix}
          \begin{pmatrix}
            \hat{x} \\ \hat{y}
          \end{pmatrix}$
          \item Clearly $ \hat{r} \cdot \hat{\theta} = 0$
          \item \scalebox{1.2}{$d\hat{l} = \delta r \hat{r} + r \delta \theta \hat{\theta}$}
          \item the $ \nabla$ operator in curvilinear co-ordinates is given by: \\ \\
          \scalebox{1.2}{$ \nabla = \frac{\partial}{\partial r} \hat{r} + \frac{1}{r} \frac{\partial}{\partial \theta} \hat{\theta}$}
          \item \scalebox{1.2}{$\hat{\textit{v}} = \frac{dr}{dt} \hat{r} + \frac{d \theta}{dt} \hat{\theta}$}
          \item \scalebox{1.2}{\textit{$\vec{a} = \big(\ddot{r}-\dot{\theta}^{2}r\big) \hat{r} + \big( r\ddot{\theta} + 2\dot{r}\dot{\theta} \big)\hat{\theta}$}}
  \end{itemize}
\end{itemize}

\section{Recorded Lecture: 3D Curvilinear Systems}
\begin{itemize}
  \item  Exercise: Do the 2D Calculations assuming the centre at a point on the diameter and taking the ranges of $\phi$ and $r$ accordingly.
  \item \scalebox{1.2}{$\frac{d\hat{\theta}}{dt} = -\dot{\theta}\hat{r}$ \\ $\frac{d\hat{r}}{dt} = \dot{\theta}\hat{\theta}$}
  \item Challenge 1; Show Kepler's Second Law implies $a_{\theta} = 0$.(Second Law is about areal velocity being constant)
  \item Challenge 2: Kepler's Law essentially applies that $\vec{F}(\vec{r}) = f(s)\hat{s}$, where s is basically the magnitude of r, the radial distance
  Using this show that $f(r) = \frac{-k}{r^{2}}$. Using the fact that the Planet is at one of the focii.
  \item The Volume Element in 3D is : $r^{2}\sin(\theta)drd\theta$
  \item The Area Elements taking two elements could be calculated easily by writing them in the vector form with differential increments in all the different co-ordinates and then taking cross products, Similarly in Cylindrical Co-ordinates. (Similar is the process to calculate Volume!)

  \item These differential increments in the different quantities could be calculated by simply taking $dx\hat{x} + dy\hat{y} + dx\hat{z}$ and trying to putting all co-ordinates in their new form. (Exercise: Do so for Spherical Co-ordinates)

  \item By taking dot of these differential increments with gradient we should get $\Delta f$, and using that we calculate the Gradient in these co-ordinates. (By Making sure they adhere to Chain Rule!)


  \item \textbf{Formulae:}

  \begin{itemize}
    \item \scalebox{1.2}{ $\sum_{i = 1}^{n}{h_{i}du_{i}\hat{u_i}}$} \\ \\
    where the parameterization has $n$ co-ordinates $\{u_i\}$ for $i = 1....n$ and $h_i$ are the various coefficient depending on the way we try to re-parameterize from the Cartesian System, building from here only, we have:
    \item \scalebox{1.2}{$\nabla = \sum_{i = 1}^{n}{\frac{1}{h_i}\frac{\partial}{\partial u_{i}}\hat{u_i}}$}
  \end{itemize}

\end{itemize}
\end{document}
