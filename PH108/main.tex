\documentclass{article}
\usepackage[utf8]{inputenc}
\usepackage{graphicx}

\addtolength{\oddsidemargin}{-.875in}
\addtolength{\evensidemargin}{-.875in}
\addtolength{\textwidth}{1.75in}
\usepackage{amsmath}
\addtolength{\topmargin}{-1in}
\addtolength{\textheight}{1.75in}

\usepackage{hyperref}
\hypersetup{
    colorlinks=true,
    linkcolor=blue,
    filecolor=magenta,
    urlcolor=cyan,
}

\title{PH108}
\author{Shubh Kumar}
\date{IITB, Spring Semester 2021}

\begin{document}

\maketitle

\section{Lecture 1 : Introduction}

\begin{itemize}
    \item G and K, the constants of Gravitation and Electrostatic Forces are called Coupling Constants when taken together

    \item A Physical Law is one which is true for observers from all \textit{Frames of References}.

    \item Strong Force holds the Nucleus Together. Weak Force is the one responsible for Radioactivity
\end{itemize}

\section{Formula from Recorded Lecture}
\begin{itemize}
    \item Pertaining to the $\nabla$ operator:
        \begin{itemize}
            \item $\nabla (fg) = f\nabla g + g\nabla f$
            \item $\nabla (\vec{A} \cdot \vec{B}) = \vec{A} \times ( \nabla \times \vec{B}) + \vec{B} \times ( \nabla \times \vec{A})+
            (\nabla \cdot \vec{A})\vec{B} + (\nabla \cdot \vec{B})\vec{A}$
        \end{itemize}




\item Those Pertaining to Divergence ($\nabla \cdot$):
        \begin{itemize}
            \item $\nabla \cdot (f\vec{A}) = f(\nabla \cdot \vec{A}) + \vec{A} \cdot (\nabla f)$
            \item $\nabla \cdot (\vec{A} \times \vec{B}) = \vec{B} \cdot (\nabla \times \vec{A})-\vec{A} \cdot (\nabla \times \vec{B})$
        \end{itemize}
\item Those Pertaining to Curl ($\nabla \times$):
\begin{itemize}
    \item $\nabla \times (f\vec{A}) = f(\nabla \times \vec{A}) - \vec{A} \times (\nabla f)$
    \item $\nabla \times (\vec{A} \times \vec{B}) = (\vec{B}. \nabla)\vec{A} + \vec{A} (\nabla \cdot \vec{B})-(\vec{A}. \nabla)\vec{B}-\vec{B} (\nabla \cdot \vec{A})$
\end{itemize}

\item \textbf{Curl of Curl:} $\nabla (\nabla \cdot \vec{A})- \nabla^{2}\vec{A}$
\end{itemize}

\section{Lecture 2: }
Nothing Extra as such, apart form Lecture Slides!

\section{Recorded Lecture 2: Flux, Gauss's Divergence Theorem, Stokes', Curvilinear Co-ordinates in a plane}
\begin{itemize}
  \item The $\hat{r}$ depends implicity on $\theta$. (Although not explicitly!)
  \item Similar to the way in which, we used to take the norm of the area vector by taking its deriavtive with the numbers, which
  define it, we could do something similar with length as well!
  \item The $\nabla$ operator takes a different form when we consider Curvilinear co-ordinates in a plane,as conventionally its defined as the change of the scalar function w.r.t the changes in the x and y co-ordinates only, such that when we take a line-integral taking the distance appropriately in Curvilinear co-ordinates then we could get the right answer.
  \item \textbf{Formulae:}
  \begin{itemize}
    \item $\begin{pmatrix}
              \hat{r} \\
              \hat{\theta}
          \end{pmatrix} =
          \begin{pmatrix}
            \cos (\theta) && \sin (\theta) \\
            -\sin (\theta) && \cos (\theta)
          \end{pmatrix}
          \begin{pmatrix}
            \hat{x} \\ \hat{y}
          \end{pmatrix}$
          \item Clearly $ \hat{r} \cdot \hat{\theta} = 0$
          \item \scalebox{1.2}{$d\hat{l} = \delta r \hat{r} + r \delta \theta \hat{\theta}$}
          \item the $ \nabla$ operator in curvilinear co-ordinates is given by: \\ \\
          \scalebox{1.2}{$ \nabla = \frac{\partial}{\partial r} \hat{r} + \frac{1}{r} \frac{\partial}{\partial \theta} \hat{\theta}$}
          \item \scalebox{1.2}{$\vec{\textit{v}} = \frac{dr}{dt} \hat{r} + r\frac{d \theta}{dt} \hat{\theta}$}
          \item \scalebox{1.2}{\textit{$\vec{a} = \big(\ddot{r}-\dot{\theta}^{2}r\big) \hat{r} + \big( r\ddot{\theta} + 2\dot{r}\dot{\theta} \big)\hat{\theta}$}}
  \end{itemize}
\end{itemize}

\section{Recorded Lecture: 3D Curvilinear Systems}
\begin{itemize}
  \item  Exercise: Do the 2D Calculations assuming the centre at a point on the diameter and taking the ranges of $\phi$ and $r$ accordingly.
  \item \scalebox{1.2}{$\frac{d\hat{\theta}}{dt} = -\dot{\theta}\hat{r}$ \\ $\frac{d\hat{r}}{dt} = \dot{\theta}\hat{\theta}$}
  \item Challenge 1; Show Kepler's Second Law implies $a_{\theta} = 0$.(Second Law is about areal velocity being constant)
  \item Challenge 2: Kepler's Law essentially applies that $\vec{F}(\vec{r}) = f(s)\hat{s}$, where s is basically the magnitude of r, the radial distance
  Using this show that $f(r) = \frac{-k}{r^{2}}$. Using the fact that the Planet is at one of the focii.
  \item The Volume Element in 3D is : $r^{2}\sin(\theta)drd\theta$
  \item The Area Elements taking two elements could be calculated easily by writing them in the vector form with differential increments in all the different co-ordinates and then taking cross products, Similarly in Cylindrical Co-ordinates. (Similar is the process to calculate Volume!)

  \item These differential increments in the different quantities could be calculated by simply taking $dx\hat{x} + dy\hat{y} + dx\hat{z}$ and trying to putting all co-ordinates in their new form. (Exercise: Do so for Spherical Co-ordinates)

  \item By taking dot of these differential increments with gradient we should get $\Delta f$, and using that we calculate the Gradient in these co-ordinates. (By Making sure they adhere to Chain Rule!)


  \item \textbf{Formulae:}

  \begin{itemize}
    \item \scalebox{1.2}{ $\sum_{i = 1}^{n}{h_{i}du_{i}\hat{u_i}}$} \\ \\
    where the parameterization has $n$ co-ordinates $\{u_i\}$ for $i = 1....n$ and $h_i$ are the various coefficient depending on the way we try to re-parameterize from the Cartesian System, building from here only, we have:
    \item \scalebox{1.2}{$\nabla = \sum_{i = 1}^{n}{\frac{1}{h_i}\frac{\partial}{\partial u_{i}}\hat{u_i}}$}  \\
    \\
    This is only the $\nabla$ operator, the curl and divergence operators have some different features!
  \end{itemize}

\end{itemize}

\section{Tutorial 1: $\nabla$ operator, Curl, Divergence, Stokes's, Gauss's, Planar Curvilinear Co-ordinates}
\begin{itemize}
  \item The divergence at a point of stable equilibrium must be -ve.
  \item But, if the equilibrium is stable, than the divergence needs to be negative.
  \item If divergence is zero, it could also mean that the direction in which the field is outwards is equal to those, for which its inward.
  \item Similarly, If the Equilibrium is unstable, then the divergence needs to be positive.
  \item
\end{itemize}

\section{Live Lecture: Discussion on L2,L3 on Curvilinear Co-ordinates}
\begin{itemize}
  \item A Formula which I missed from the previous  section: \\
  $\begin{pmatrix}
    \dot{\hat{r}} \\
    \dot{\hat{\theta}}
  \end{pmatrix}
 = \dot{\theta}
 \begin{pmatrix}
  -\sin \theta && \cos \theta \\
  -\cos \theta && - \sin \theta
 \end{pmatrix}
 \begin{pmatrix}
   {\hat{x}} \\
   {\hat{y}}
 \end{pmatrix}
 = \dot{\theta}
 \begin{pmatrix}
   0 && 1 \\
   -1 && 0
 \end{pmatrix}
 \begin{pmatrix}
   \hat{r} \\
   \hat{\theta}
 \end{pmatrix}
  $

  \item The Transformation Matrix in the last part is Skew-Symmetric(point to be noted!)
  \item From one orthonormal basis to another orthonormal basis, Unitary Transformation is involved.
  \item Let's see what is it so:
    \begin{itemize}
      \item First thing, if I have any unit vector, then its orthonormal to its differential
      \item Now, If I have the transformaion matrix between two orthonormal basis as $mathbf{M}$, then $\mathbf{MM}^{T} = I$ and if we differentiate this matrix, we find:
      $\mathbf{\dot{M}M^T + M\dot{M}^T = 0}$ which means $\mathbf{MM}^{T}$ is skew-symmetric!
    \end{itemize}
  \item \textbf{Levi-Civita Symbol System:}
  \begin{itemize}
    \item If an index is repeated, then summation is automatically implied.
    Like, the minute I have \textit{j} and \textit{k}, it in itself implies a summation!
    \item For more info head \href{http://www.physics.usu.edu/Wheeler/ClassicalMechanics/CMnotesLeviCivita.pdf}{here}.
  \end{itemize}

\end{itemize}

\section{Recorded Lecture 4: Curl, Divergence in Curvilinear Co-ordinates and Multi-Dimensional Dirac-Delta Function}

\begin{itemize}
  \item While Applying the Divergence Theorem, or any theorem, If we have a point where the function isn't defined, then that could lead to trouble.
  \item We define the Dirac-Delta as: \\
  \begin{center}
    $  \delta (x-x_0) = \begin{cases}
                      1 & x = x_0 \\
                      0 & otherwise \\
                  \end{cases}$
  \end{center}

  Also: \\
  \begin{center}
    $  \int_{a}^{b}{f(x) \delta (x-x_0) dx} = \begin{cases}
                      f(x_0) & a < x_0 < b \\
                      0 & otherwise \\
                  \end{cases}$
  \end{center}

  \item Using the Dirac-Delta Function, we define the Divergence,
  \begin{center}
    $ \nabla \cdot \big( \frac{\hat{r}}{r^2} \big) = 4 \pi \delta ^{3} (\vec{r}) = 4 \pi \delta (x) \delta (y) \delta (z)$
  \end{center}

  \item \textbf{Formulae:}
  \begin{itemize}
    \item  \scalebox{1.2}{$\nabla \cdot \vec{F} = \frac{1}{h_{1}h_{2}h_{3}} \sum_{i = 1}^{n}{\frac{\partial (h_{1} h_{2}...h_{i-1} h_{i+1}...h_{n} F_i)}{\partial u_i}}$}   \\
    Got by calculating Flux, taking two sides at a time and then removing the differential
    \item \scalebox{1.2}{$\nabla \times \vec{F} = \frac{1}{h_2 h_3} \Big[ \frac{\partial h_3 F_3}{\partial u_2} - \frac{\partial h_2 F_2}{\partial u_3} \Big] \hat{u_1} + \frac{1}{h_3 h_1} \Big[ \frac{\partial h_1 F_1}{\partial u_3} - \frac{\partial h_3 F_3}{\partial u_1} \Big] \hat{u_2} + \frac{1}{h_1 h_2} \Big[ \frac{\partial h_2 F_2}{\partial u_1} - \frac{\partial h_1 F_1}{\partial u_2} \Big] \hat{u_3} $} \\
    This is derived by taking area in three cases, varying two and keeping the thrid co-ordinate fixed in each, like this by using the Stokes Theorem, we can get the value of the Curl in that direction by calculating the Line Integral, in the differential form ,and then generalizing by exploiting that they permutate.
    \item
  \end{itemize}
\end{itemize}

\section{Recorded Lecture 5: Summary/Re-explaination of Dirac-Delta Function and Helmholtz's Theorem}

\begin{itemize}
  \item $\int_{- \infty}^{\infty}{e^{i(k-k_0)}} = \delta (k-k_0)$
  \item \textbf{Helmholtz's Theorem:} It comes as a way to get the Force if we are given its Divergence, Curl as well as Boundary Conditions for it.  \\[0.125in]
  \begin{center}
    If Suppose, $D(\mathbf{r}) = \nabla \cdot \vec{F}(\mathbf{r})$ and $C(\mathbf{r}) = \nabla \times \vec{F}(\mathbf{r})$, then we define: \\[0.25in]
        $ U(\mathbf{r}) = \frac{1}{4 \pi} \int_{}^{}{\frac{D(\mathbf{r})}{|\mathbf{r - r'}|}} dV' $ and  \\[0.125in]
        $ W(\mathbf{r}) = \frac{1}{4 \pi} \int_{}^{}{\frac{C(\mathbf{r})}{|\mathbf{r - r'}|}} dV' $  \\[0.125in]
        Then we may write: \\~\\
        $\vec{\mathbf{F}} = -\nabla U(\mathbf{r}) + \nabla \times W(\mathbf{r}) $
  \end{center}
\end{itemize}

\section{Live Lecture: Discussion on Curl, Divergence in Curvilinear Co-ordinates}

\begin{itemize}
  \item The Dirac-Delta's Function's complete use in QM can be visualized as: The Delta function at different $x_0$ are all eigenfunctions of the position operator and hence, the position could be expressed as a linear combination of all these basis functions. Hence, While we are integrating to calculate the expectation value of Position, then it is actually an integration over Dirac-Delta Functions, each multiplied with a function $f(x_0)$.

\end{itemize}

\section{Tutorial 2}

\begin{itemize}
  \item The Divergence, Curl expressions obtained in 3D are such that they don't compensate for the $dV$ or $dA$ terms which appear in The various Theorems.
  \item The $d\vec{S}$ vector is obtained not as a unit vector, but the proper area vector of that infinitesimal area.
  \item if $\vec{a} \cdot \vec{c} = \vec{b} \cdot \vec{c}$ for all $\vec{c}$, then $\vec{a} = \vec{b}$.
  \item Circular Permutations in Triple Scalar Products: An important Technique! Used in Q4 and Q7
  \item 
\end{itemize}


\end{document}
