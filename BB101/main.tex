\documentclass{article}

\addtolength{\oddsidemargin}{-.875in}
\addtolength{\evensidemargin}{-.875in}
\addtolength{\textwidth}{1.75in}

\addtolength{\topmargin}{-1in}
\addtolength{\textheight}{1.75in}


\usepackage{hyperref}
\hypersetup{
    colorlinks=true,
    linkcolor=blue,
    filecolor=magenta,
    urlcolor=cyan,
}
\title{\textbf{BB101}}
\author{Shubh Kumar}
\date{Spring Semester 2021}

\begin{document}

\maketitle

\section{Introduction and Tour of Cell}
\begin{itemize}
    \item To tackle big Bio Challenges we need inter-disciplinary applications to develop technology
    \item Nature gives us insights for better efficiency in technological designs eg.
    \begin{enumerate}
        \item Kingfisher catching fishes from water inspired Bullet Trains
        \item Lotus Leaves have crevices, i.e. rough leaf surface, through which water which trap air and becomes hydrophobic so that water flows out, this inspired Waterproofing by company Green Shield Construction Chemicals for lotus effect in paint.
        \item Architecture, termites' homes' structures are known to have cooling temperature, as it cools it through passive cooling, air comes from holes and hot air goes away from top. Mick Pierce designed East Gate Building in Zimbabwe inspired from this uses only 10 per. for electricity
        \item  Dolphins recognize each other by the frequency of calls of their mates by gauging the presence of specific frequencies in them and their scattering properties, EvoLogics used it to make underwater communication and sonar underwater robotics, currently there in indian ocean to detect catastrophe.
        \item Namib Desert Beetle collects water droplets from morning fog and let moisture roll as its meshy,into its mouth same concept was used by an IIT -B Student to create a fence-like mesh could be used in arid conditions to collect water from fog in the morning.
    \end{enumerate}

    \item Google Brain, Google X Smart Contact lens(shaped like eye, as Glucose from tears enter it diabetes could be detected), IBM in Computational Biology, IBM Watson working on BlockChain, AI and also used for AI assistance in stuff like COVID-19

    \item \textbf{IIT-B Research:} Based Portable Diagnostic System for urine analysis and sugar diagnosis, Knee based Prosthetic, Mobile Apps to diagnose Sickle Cell Anaemia exploiting differential RBC count.


    \item \textbf{IITB COVID Projects:} Helmet Patient Interface, Portable UVC Germicidal Unit, Self-disinfecting Masks, Tapestry

    \item \textbf{Proteomics at IIT-B for COVID:} To decide from protein analysis whether or not a case could become serious
    \item \textbf{Biological Complexity:}Around 2000, the major Genome Project was processed and now we know a lot about number and types of genes in various organisms, Gene Numbers are more or less same and is no indication of advancement.

    \item \textbf{Central Dogma of Life:} Genome-> Chromosomes-> DNA (->Transcription(RNA)-> Translation(Protein)). Genes remain static throughout life, the active component RNA is transcribed, which leads to Proteins. Proteins thus are dynamic and help us to respond to drugs, microbes etc.

    \item \textbf{Life Processes:} Energy Processing, Order, Regulation, Evolutionary Adaption, Environmental Response, Reproduction

    \item \textbf{Cell and its Properties:}
     \begin{enumerate}
         \item Light Microscopy is used for live cells, whereas Electron Microscopy is used mostly for dead cells. Resolution varies inversely with Wavelength.
         \item \textbf{Prokaryotic Cells:} Earliest Form of life, This is the first life which has evolved through many conditions over the billions of years.
         \begin{itemize}
             \item Cell wall and capsule composition needs to be studied properly as it decided which form of substances could be allowed to enter our cell
             \item Fimbri is hair like substances to help the cell move through various things.
             \item The Flagella helps in motion and different activity. Composed of Motor, Hook and Filament, differs from Eukaryotes
             \item Free-floating DNA Material(Nucleoid), lacks nucleus
             \item Lacks compartments, no compartments
             \item Bacterial Prokaryotes also have some extra chromosomal bacterial known as plasmid(double stranded, covalently closed, circular genetic material, which replicate and inherit independent of Bacterial Chromosomes which are the other component of Genetic Material in Bacteria) used for recombination activities
             \item Bacterial Cell Wall has peptidoglycan layer which is sugar cross linked with small polypeptides, helps bacteria to extend on surface, Their thickness and distribution could help decide whether it is gram positive or gram negative bacteria, Gram Positive have thick layer, Gram Negative have less PeptidoGlycan and also have an extra membrane. The Gram Negative are actually more complex due to  having 3 sandwiched layer with thin peptidoglycan but also have an outer membrane of lipopolysaccharide, This aids Staining Method to differentiate.
             \item \textbf{Staining Method} Comes out when a scientist was figuring ways to diagnose and treat patients.
             \item \textbf{Gram Positive/Gram Negative helps in Diagnosis and Treatment:}Penicillin helps peptidoglycan crosslinking especially in Gram Positive Bacteria

             \item \textbf{Gram Staining Procedure:}  Sterilize and Cool loop, take a loop full of culture and make a smear on a clean slide, Then let it heat fixate to slide, then add crystal violet, then wash of stain and add Gram Iodine, which forms complex with crystal violet and gets stuck to wall. then when we wash with acetone-alcohol(ethanol!) decolourizer the gram  positive stay violet whereas gram negative are decolourized and later colored with safranin. After that we'll wash off the safranin and let smear be air-dried. Then add emulsion oil and observe under microscope.
         \end{itemize}
         \item \textbf{Eukaryotic Cell}: True Nucleus, With all genetic material inside, and also has many compartments.
         \begin{itemize}
             \item Few Genes in Mitochondria as well
             \item Ribosomes are organelle which synthesize the Proteins
             \item Peroxisomes are organelles to catalyze Hydrogen Peroxide.
             \item Micro-villi are there to help movement
             \item Micro-tubules which contain Centrosomes
             \item Two types: Animal and Plant Cells
             \item Both are mostly same, but some differences such as Cell Wall, Plasmodesmata(which provides connectivity with other cells), Plasma Membrane is also there, Central Vacuole for waste materials, Chloroplast is there specific to plants(also having unique genes)[All these are there in Plants but not in animals]
             \item Cytoskeleton: Microtubules etc. are part of it, provides structure to cell and thus different organelles.
             \item Plasma Membrane also helps taking signals from outside.
             \item  Golgi Apparatus also helps in exporting waste out of the cell
             \item Plastids is the superset of Chloroplasts, ChromoPlasts and LeucoPlasts
             \item \textbf{Lysosome:}
             \item \textbf{Glyoxysome:}
         \end{itemize}
     \end{enumerate}

     \item \textbf{Histones:} Characteristic proteins which encode the DNAs in Archaea and Euckaryotes. It is a Homologous Protein.

\end{itemize}

\section{Lecture 1's Interactive Session}
\begin{itemize}
    \item Archaea are capable of growing in extreme conditions such as high temperature, salty environment etc.
     \item Eukaryotes are known to evolve through various symbiotic relationships between archaea and other bacteria.
     \item We can play with BioInformatics at \href{https://www.ncbi.nlm.nih.gov/}{By Getting the Protein Sequences from here} and \href{https://www.ebi.ac.uk/Tools/msa/clustalo/}{and Here to compare proteins.}
     \item Safranin Dye is cationic which bonds with the DNA in Cell wall and its the DNA which gives color like that
     \item We Heat fix the smear before observing, so that it denatures the proteolytic enzymes which could cause cell lysis(i.e. Enzymes breaking the cell membrane/Wall)
\end{itemize}

\section{1st Tutorial Section}
    \begin{itemize}
        \item A and T are joined by 2 hydrogen bonds whereas C and G and joined by 3 Hydrogen Bonds
    \end{itemize}

\section{2nd Lecture}
\begin{itemize}
    \item In Gram Postitive Bacteria, We find that the 20-80 nm homogenous layer of peptidoglycan forms the cell Wall
    \item It contains Lipoteichoic acid which is connected either to the PeptidoGlycan or the Plasma Membrane's lipids itself.
    \item There's a peri-plasmic space separating Plasma Membrane and the peptidoglycan.
    \item Cell Wall has certain Integral Proteins in addition to a bi-layer formed of Phospholipids forming the polar head of the surface, and the fatty acyl chain towards the inside.
    \item In Gram Negative Bacteria: Porin Proteins are found about the outer Membrane(formed of lipopolysaccharides) which allow transfer of molecules through it.
    \item \textbf{Ribosomes in Eukaryotes:}composed of Proteins and Ribosomal RNA
    \begin{itemize}
      \item The Prokaryotic 70S Ribosomes is made of a large 50S ribosome and a small 30S Ribosome.
      \item S is Svedberg Coefficient which provides indication about the rate of sedimentation
    \end{itemize}
      \item Flagella is made of protein flaginin. It has helical structure with a sharp hook. To swim forward, it rotates in counter-clockwise. When the rotation abruptly changes to clockwise, the bacteria tumbles.
    \item Mitochondria has Porin Channels to help flow of materials from outside. It has ribosomse, ATP synthase, Cristae which are the foldings of its inner structure, Matrix which fills its inside, adn two membranes(inner and outer)
    \item \textbf{Ribosomes in Eukaryotes:}composed of Proteins and RNA, they read nuclic acid information from mRNA and convert it to Proteins.They have 80S units consisting of a large 40S subunit which binds to amino acids, and a small 28S subunit which binds to mRNA during protein synthesis
    \item \textbf{Endoplasmic Reticulum:} Rough ER is a major site of protein synthesis, while the SER synthesizes lipids, steroids, metabolisis, carbohydrates and regulates calcium concentration in muscles.
    \item \textbf{Golgi Apparatus:}Packages and Transports macromolecules, within and outside the cells

    \item Lysosome takes care of radicals in cells, and any intracellular debris.
    \item They contain Hydrolytic enzymes in sacs to do this!
    \item Peroxisomes also take care of digesting low-chain fatty acids
    \item They are single cell organisms
    \item \textbf{Endosymbiotic Origin of Double Membrane Structures in Eukaryotic Cells:} Aerobic non-photosynethic bacteria was engulfed by another prokaryotic cell and became part of it. ChloroPlast similarly came as some algae was engulfed in the cell.
    \item A proof to the previous point is that Genetic Material and Protein Synthetic Machinery is different in other Double Membrane Structures and Nucleus.
    \item 8 Histome Proteins attach to the DNA Molecules to form Nucleosome, These Nucleosome gets packed to give two different types of Chromatin
    \item Chromatids are less condensed than chromosomes. A chromosome consists of a single, double-stranded DNA molecule. Chromatids are two molecules of double-stranded DNA joined together in the center by a centromere. Chromosomes have a thin ribbon-like structure.

    \item \textbf{Cell Cycle:}Every cell must have come from the other cell , In the video part we say that
    \begin{itemize}
        \item \textbf{Mitosis:}Identical chromatids, one of which identifies as Father and another as mother(pair of homologous but non-sister chromatids)
        \item \textbf{Meiosis:}Responsible for shuffling of genes and hence variation in features
        \begin{itemize}
            \item In Prophase I in Meiosis, There's a synaptic site between the two non-sister chromatids: Chiasma from where some genes crosses over from one bunch of chromatids to the other and hence leading to shuffling of genes
            \item Then the complete thing happens just like Mitosis in Meiosis I and then Meiosis II happens which divided the Chromosomes to 2 further pieces to give 4 haploid daughter cells

        \end{itemize}
    \end{itemize}
    \item \textbf{Regulation of Cell Cycle:}Cyclins(made during G2) are a family of proteins which alongwith Cyclin-dependent kinases(made during G1). Both are produced during beginning of Reproduction, and the Cyclins get and fit into CDK as a key-lock pair, which is completed only after M-phase and after that, they dissociate, and cyclin is degenerated.
    \item \textbf{External Regulations:} Space, Nutrients, Density-Dependent Inhibiton
    \begin{itemize}
        \item Cancers cells lose dependence on these external as well as Internal factors for Signals for Proliferation.
        \item That's why even after Cancer is treated, it remains in meta-static phase in some cases and grows again after few years.
    \end{itemize}
    \item \textbf{Cell Re-programming and Cloning:}
\end{itemize}

\section{Important Stuff from Whatsapp Groups:}

\begin{itemize}
  \item The Rough E.R is continuous with Nuclear Membrane, the SER is comparatively distant.
  \item Histone forms an octamer around which DNA makes 1 and 3/4th turn to form a nucleosome(like a unit for coiling ) and
  Histone +ve charged DNA negative
So it's stuck and won't get uncoiled
\end{itemize}


\section{Lecture 3: Cellular Metabolism}
\begin{itemize}
  \item Mesophyll are a type of Cell which contain Chloroplast.
  \item They contain stomata.
  \item \textbf{Light Reaction:}Occurs in presence of Sunlight
  \begin{itemize}
    \item ATP and NADPH are produced.
    \item takes place in granum of ChloroPlast
    \item ATP and NADPH are further taken to Stroma where dark reaction/Calvin Cycle are performed.
    \item Chlorophyll absorbs Vilet-Blue and Orange-red thus, only green is transmited
    \item $H_2O$ gives $H^{+}$ reacts with $O_2$ and consumes some electrons from P680(Photosystem II or PSII), this P680 does so by absorbing light and gets excited to $P680^{+}$, these electrons are carried through a series of pigments which are plastoquinone and finally goes to CyctoChrome-B complex and then Plasto-cinin and then transferred to Photo-System I($P700^{+}$), and the reduction(happens through Ferridoxine Molecule) to $NADP^{+}$ and $H^{+}$ and in presence of $NADP^{+}$ reductase, NADPH is formed.
    \item When the Pq-Cyctochrome-Pc, that is only where using chemo-osmotic processes ATP Synthesis occurs.

    \item In certain cases, when the flow becomes cyclic and only PSI is involved, we find that NADPH isn't produced in that case.
  \end{itemize}
  \item \textbf{Dark Reaction:}Sugar/GLucose is produced (Calvin Cycle/ C3 Cycle)
  \begin{itemize}
    \item Three phases: Carboxylation, Reduction and Regeneration
    \item This gives a 3-C product, Triose Phosphate, which gives it its name.
  \end{itemize}
  \item \textbf{C4 Cycle:}In Plants with hot climate regions(some:)
  \begin{itemize}
    \item First stable compound: OAA(Oxaloacetate 4 Carbon) combines with PEP(Phosphoenol-Pyruvic Acid) which eventually becomes Malic Acid(in presence of Malonic de-hydrogenase), which further decomposes to Pyruvic Acid and $CO_2$(with the help of $H^{+}$ from NADPH breaking to NADP and $H^{+}$)[Happening in Bundle Cells].
    The Pyruvic acid re-enters mesophyll cell and helps in generating PEP.

    \item CAM(Crassulacean acid metabolism) plants have stomato closed in morning and open in night. $CO_2$ stored in Night is used in day time for photosynthesis. (Found in Plants living in Arid Conditions)

    \item Both CAM and C4 are found in hot and arid climates.
  \end{itemize}

  \item \textbf{Metabolism:}
  \begin{itemize}
    \item \textbf{Cellular Respiration (Catabolism):}
    \begin{itemize}
      \item Pyruvic Acid formed after Glycolysis, undergoes Aerobic/Anaerobic Processes depending upon the availability of $CO_2$. Aerobic leads to Krebs Cycle.
      \item \textbf{FAD, $FADH_2$:} Electron transfer agents like $NAD^{+}$ and NADH.
      \item  Energy yeild depends on what Electron Transfer Process is taken.
    \end{itemize}
    \item Cancer Metabolism: The Observations of Warburg point that Intrinsic and Extrinsic Factors, could change Biosynthesis, Bioenergetics, Redox State of cell and have correlation with Cancer.
    \item ATP generation in glycolysis is required for the cancer cells to grow and proliferate in unrestricted manner.(PKM2 shows that this pathway isn't the only source of energy for Cell Proliferation, and if its changed by Extrinsic factors then it could have profound implications.)
    \item FDG-PET test goes over the entire body and see where there is a high uptake of Glucose to detect tumours.
    \item Other Tests look at Metabolic substances(Metabolites) to detect Cancer.
    \item \textbf{Cancer Research Highlights:} Proteomic Investgation: See the grade of proteins which are present in various fluids in the brain. To appreciate the variations between the sick and healthy people. In order to get an idea about what's causing it and how to prevent it. Using PCA to see if we could separate clusters of these proteins to different categories, which could give us comprehensive inferences to the proteins' importance. We also look at their numbers and also radiology data.
    Isocitrate Dehydrogenase Mutations are quite common in Gliomas patients. Gliomas are the most basic of cancers.
  \end{itemize}
\end{itemize}

\section{Live Lecture 3}
  \begin{itemize}
    \item \textbf{Mitochondrial Diseases:} Mitochondra use process known as phosphorylation to generate energy. It takes thousands of proteins for it to function, A single one wrong could cause problems. Only 13 of the Proteins could get got from Mitochondria, other come from DNA. The Mitochondrial diseases could hamper any organ, tissue and happen anytime in life. eg. Leigh Syndrome happens in Children could lead to many problems like mental retardation and later to loss in Movement abilities, but the disease can be stopped before it passed on. Mitochondrial DNA comes from mother, if its found, it could be transplanted from healthy mothers. This technique was recently approved in UK.
    \item \textbf{Cell Cycle and Totipotency Review:}
    \begin{itemize}
      \item Morghulis gave the Model for Endosymbiotic Origin for double membrane structures.
      \item G1 phase is where DNA synthesis happens and makes sure that all conditions is favourable for the Chromosome Replication in S phase.
      \item G2 Phase covers growth, to further make sure that the cell is ready for dividing and reproduction.
      \item In Meiosis, Interphase happens only once.(Before Meiosis I)
      \item The Checkpoints are presents somewhere before the end in G1, as well as the beginning and end of M phase.
      \item Cytokninin and Gibberelin helps the totipotent cells proliferate to form the callus.
      \item When Cyclin gets conjugated with CDK, and it forms a complex MPF(Mitosis Promoting Factor) which induces phosphorylation and helps in the process of Mitosis.
      \item Cloning has also benefitted a lot from the concepts of Checkpoints in the Cell Cycle.
      \item iPS: Induced Pluripotent Stem
      \item Allele are different version of gene on corresponding loci of chromosomes(Locii of genes which contain information about a particular trait)
      \item In Chromosome Duplication, the single-stranded Chromosome when they replicate, two of them remain attached at their centromere(with kinetochore proteins over them!), which are known as Chromatids.
      \item Prometaphase happens in Meiosis I?
    \end{itemize}
    \item Endomembrane Systems include ERs(SER and RER) besides Golgi Apparatus.
    \item
  \end{itemize}

\end{document}
